\documentclass[a4paper,10pt,twocolumn]{article}
\usepackage{algorithmicx}
\usepackage{algpseudocode}
\date{ }
\title{My First Article in \LaTeX}
\author
{
\begin{tabular}[t]{c@{\extracolsep{30mm}}c@{\extracolsep{30mm}}c}
{\it Author-1} & {\it Author-2} & {\it Author-3}\\
Affiliation & Affiliation & Affiliation\\
Address & Address & Address\\
e-mail & e-mail & e-mail\\
\end{tabular}
}
\begin{document}
\maketitle
%
\begin{abstract}
Abstract of the article … Abstract of the article …
\end{abstract}
%
\section{Introduction}
Sorting algorithms play a fundamental role in computer science, underpinning countless applications from database management to scientific computing and machine learning. Their efficiency directly impacts the speed and performance of these systems, making the search for optimal sorting algorithms an ongoing pursuit. This study delves into two prominent sorting algorithms: Quicksort and Timsort, exploring their respective strengths, weaknesses, and unique approaches to organizing data.
\subsection{Importance of Sorting Algorithms:}
The ability to efficiently organize and search through large datasets is vital in almost every aspect of modern computing. Sorting algorithms provide the foundation for this capability by arranging elements in a specific order, enabling rapid retrieval and analysis. From searching for keywords in web documents to analyzing financial transactions, sorting algorithms underpin countless applications that shape our daily lives.
\subsection{The Need for Different Approaches:}

While the goal of sorting remains consistent, the nature of data and the specific requirements of different applications necessitate a variety of algorithms. No single sorting technique excels in all scenarios. Quicksort, for instance, boasts impressive average-case time complexity, making it a popular choice for general-purpose sorting tasks (Cormen et al., 2009). However, its worst-case performance can be significantly slower, particularly for already-sorted data.

\subsection{Introducing Quicksort and Timsort:}
Quicksort, conceived by Tony Hoare in 1959, is a divide-and-conquer algorithm that relies on partitioning the data around a chosen pivot element (Hoare, 1961). This recursive process continues until all sub-arrays are sorted. Quicksort's simplicity and efficiency have made it a widely used algorithm, although its worst-case performance can be a concern.\\\\
Timsort, developed by Tim Peters in 2002, takes a more nuanced approach (Peters, 2002). It combines elements of merge sort and insertion sort, leveraging existing order within the data (runs) and efficiently merging them to achieve overall sortedness. Timsort's hybrid nature and adaptability make it particularly well-suited for real-world data, often exhibiting superior performance compared to Quicksort.

\subsection{Objectives of this Study:}
This study aims to provide a comprehensive comparison of Quicksort and Timsort, analyzing their respective advantages and limitations. We will explore the theoretical underpinnings of each algorithm, examining their time and space complexity in various scenarios. Additionally, we will implement both algorithms and conduct empirical evaluations on diverse datasets to compare their practical performance.
\\\\
Through this analysis, we hope to gain a deeper understanding of the trade-offs inherent in each approach and to provide insights into the selection of an appropriate sorting algorithm for different applications.

\section{Background Study}
Sorting algorithms have fascinated computer scientists for decades, with countless solutions born from the quest for efficient data organization. This background study delves into two prominent contenders: Quicksort and Timsort, dissecting their algorithms, analyzing relevant theories, and exploring prior research to lay the foundation for a thorough comparison.
\\
\bf{Quicksort:}\\
\textit{\underline{algorithm}}
\begin{algorithmic}[1]
\Function{Quicksort}{$A, p, r$}
\If{$p < r$}
\State $q \gets$ \Call{Partition}{$A, p, r$}
\State \Call{Quicksort}{$A, p, q-1$}
\State \Call{Quicksort}{$A, q+1, r$}
\EndIf
\EndFunction
\end{algorithmic}
\textit{partition}
\begin{algorithmic}[1]
\Function{Partition}{$A, p, r$}
\State $x \gets A[r]$
\State $i \gets p-1$
\For{$j \gets p$ to $r-1$}
\If{$A[j] \leq x$}
\State $i \gets i+1$
\State \Call{Swap}{$A[i], A[j]$}
\EndIf
\EndFor
\State \Call{Swap}{$A[i+1], A[r]$}
\State \Return $i+1$
\EndFunction
\end{algorithmic}
\textit{swap}
\begin{algorithmic}[1]
\Function{Swap}{$a, b$}
\State $temp \gets a$
\State $a \gets b$
\State $b \gets temp$
\EndFunction
\end{algorithmic}

\underline{\bf {Algorithm BreakDown:}}\\
Quicksort is a divide-and-conquer algorithm that recursively partitions the data around a chosen pivot element. The partitioning process involves rearranging the array such that all elements less than the pivot are to its left and all elements greater than the pivot are to its right. The pivot's final position is returned, and the algorithm is applied to the sub-arrays on either side of the pivot. This process continues until the entire array is sorted.\\\\
\underline{\bf {Time Complexity:}}\\
Quicksort's time complexity is dependent on the partitioning process, which is itself dependent on the choice of pivot. The best-case scenario occurs when the pivot is the median element, resulting in two sub-arrays of equal size. In this case, the partitioning process takes linear time, and the recurrence relation is given by:
\begin{equation}
T(n) = 2T(n/2) + \Theta(n)
\end{equation}
The solution to this recurrence relation is $T(n) = \Theta(n\log n)$, which is the best-case time complexity of Quicksort.\\\\
The worst-case scenario occurs when the pivot is the smallest or largest element, resulting in one sub-array of size $n-1$ and another of size $0$. In this case, the partitioning process takes quadratic time, and the recurrence relation is given by:
\begin{equation}
T(n) = T(n-1) + \Theta(n)
\end{equation}
The solution to this recurrence relation is $T(n) = \Theta(n^2)$, which is the worst-case time complexity of Quicksort.\\\\
The average-case time complexity of Quicksort is $\Theta(n\log n)$, which is the same as its best-case time complexity. This is because the probability of choosing the median element as the pivot is $\frac{1}{n}$, and the recurrence relation is given by:
\begin{equation}
T(n) = T(n/2) + T(n/2) + \Theta(n)
\end{equation}
The solution to this recurrence relation is $T(n) = \Theta(n\log n)$, which is the average-case time complexity of Quicksort.\\\\
\underline{\bf {Space Complexity:}}\\
Quicksort's space complexity is dependent on the partitioning process, which is itself dependent on the choice of pivot. The best-case scenario occurs when the pivot is the median element, resulting in two sub-arrays of equal size. In this case, the partitioning process takes linear space, and the recurrence relation is given by:
\begin{equation}
S(n) = 2S(n/2) + \Theta(n)
\end{equation}
The solution to this recurrence relation is $S(n) = \Theta(n)$, which is the best-case space complexity of Quicksort.\\\\
The worst-case scenario occurs when the pivot is the smallest or largest element, resulting in one sub-array of size $n-1$ and another of size $0$. In this case, the partitioning process takes quadratic space, and the recurrence relation is given by:
\begin{equation}
S(n) = S(n-1) + \Theta(n)
\end{equation}
The solution to this recurrence relation is $S(n) = \Theta(n^2)$, which is the worst-case space complexity of Quicksort.\\\\
The average-case space complexity of Quicksort is $\Theta(n)$, which is the same as its best-case space complexity. This is because the probability of choosing the median element as the pivot is $\frac{1}{n}$, and the recurrence relation is given by:
\begin{equation}
S(n) = S(n/2) + S(n/2) + \Theta(n)
\end{equation}
The solution to this recurrence relation is $S(n) = \Theta(n)$, which is the average-case space complexity of Quicksort.\\\\




\section{Results and Analysis}
Introduction to the work … Introduction to the work …

\section{Conclusion}
Introduction to the work … Introduction to the work …

\bibliographystyle{plain}
\bibliography{mybib2}

\end{document}
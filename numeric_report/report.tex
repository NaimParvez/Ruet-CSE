
\documentclass[a4paper,12pt]{article}
\usepackage{geometry}
\geometry{margin=1in}
\usepackage{graphicx}
\usepackage{listings}
\usepackage{amsmath}
\usepackage{float}

\begin{document}

\title{Lab Report: Numerical Methods}
\date{}
\maketitle

\section*{Lab Report 1: Bisection and False Position Methods for Root Finding}

\subsection*{i) Name of the Experiment}
Bisection and False Position Methods for Root Finding

\subsection*{ii) Theoretical Background}
\subsubsection*{Bisection Method:}
The Bisection method is a numerical technique for finding the root of a continuous function within a specified interval. It operates on the principle of intermediate value theorem and repeatedly narrows down the interval containing the root. The midpoint of the interval is calculated as:
\[ x_{\text{mid}} = \frac{a + b}{2} \]
where \(a\) and \(b\) are the endpoints. The method continues until the interval becomes sufficiently small or a predetermined tolerance is achieved.

\subsubsection*{False Position Method:}
The False Position method combines aspects of linear interpolation and the bisection method. It updates the interval based on linear interpolation, and the new estimate \(x\) is given by:
\[ x = \frac{a \cdot f(b) - b \cdot f(a)}{f(b) - f(a)} \]
This process is repeated until convergence or the maximum number of iterations is reached.

\subsection*{iii) Source Code}
\begin{lstlisting}[language=C++]
// ... [The source code for bisection and false position methods]
\end{lstlisting}

\subsection*{iv) Output (Screenshot)}
\begin{figure}[H]
    \centering
    %\includegraphics[width=0.8\textwidth]{bisection_false_position_output.png}
    \caption{Output for Bisection and False Position Methods}
\end{figure}

\subsection*{v) Discussion}
[Discuss the results, convergence, and limitations of the bisection and false position methods.]

\newpage

\section*{Lab Report 2: Newton's Forward and Backward Interpolation}

\subsection*{i) Name of the Experiment}
Newton's Forward and Backward Interpolation

\subsection*{ii) Theoretical Background}
\subsubsection*{Forward Newton Interpolation:}
Forward Newton interpolation is a method to approximate values between data points using a forward-divided difference scheme. The formula for the interpolated value \(f(x)\) is:
\[ f(x) \approx y_0 + (x - x_0)\Delta y_0 + \frac{(x - x_0)(x - x_1)}{2!}\Delta^2 y_0 + \ldots \]
This method is suitable when data points are arranged in increasing order of the independent variable.

\subsubsection*{Backward Newton Interpolation:}
Backward Newton interpolation, on the other hand, uses a backward-divided difference scheme. The formula for \(f(x)\) is similar but works with data points arranged in decreasing order of the independent variable.

\subsection*{iii) Source Code}
\begin{lstlisting}[language=C++]
// ... [The source code for Newton's Forward and Backward Interpolation]
\end{lstlisting}

\subsection*{iv) Output (Screenshot)}
\begin{figure}[H]
    \centering
   % \includegraphics[width=0.8\textwidth]{newton_interpolation_output.png}
    \caption{Divided Differences Table for Newton's Forward and Backward Interpolation}
\end{figure}

\subsection*{v) Discussion}
[Discuss the significance of Newton's interpolation methods and compare the results obtained from forward and backward interpolation.]

\newpage

\section*{Lab Report 4: Numerical Integration using Trapezoidal, Simpson's One-Third, and Three-Eighths Rules}

\subsection*{i) Name of the Experiment}
Numerical Integration using Trapezoidal, Simpson's One-Third, and Three-Eighths Rules

\subsection*{ii) Theoretical Background}
\subsubsection*{Trapezoidal Rule:}
The Trapezoidal Rule is a numerical integration technique that approximates the definite integral of a function. The formula for the Trapezoidal Rule is:
\[ \int_{a}^{b} f(x) \,dx \approx \frac{h}{2} \left[ f(a) + 2\sum_{i=1}^{n-1} f(x_i) + f(b) \right] \]
where \(h\) is the width of each subinterval.

\subsubsection*{Simpson's One-Third Rule:}
Simpson's One-Third Rule provides a more accurate approximation by using quadratic polynomials over two adjacent subintervals. The formula is:
\[ \int_{a}^{b} f(x) \,dx \approx \frac{h}{3} \left[ f(a) + 4\sum_{i=1}^{n-1} f(x_{2i-1}) + 2\sum_{i=1}^{n-1} f(x_{2i}) + f(b) \right] \]

\subsubsection*{Simpson's Three-Eighths Rule:}
Simpson's Three-Eighths Rule is an extension of Simpson's Rule for non-multiple-of-three intervals. The formula is:
\[ \int_{a}^{b} f(x) \,dx \approx \frac{3h}{8} \left[ f(a) + 3\sum_{i=1}^{n-1} f(x_{3i-2}) + 3\sum_{i=1}^{n-1} f(x_{3i-1}) + 2\sum_{i=1}^{n-1} f(x_{3i}) + f(b) \right] \]

\subsection*{iii) Source Code}
\begin{lstlisting}[language=C++]
// ... [The source code for numerical integration using Trapezoidal, Simpson's One-Third, and Three-Eighths Rules]
\end{lstlisting}

\subsection*{iv) Output (Screenshot)}
\begin{figure}[H]
    \centering
    %\includegraphics[width=0.8\textwidth]{numerical_integration_output.png}
    \caption{Output for Numerical Integration using Trapezoidal, Simpson's One-Third, and Three-Eighths Rules}
\end{figure}

\subsection*{v) Discussion}
The lab report now provides a comprehensive overview of three numerical integration methods: Trapezoidal Rule, Simpson's One-Third Rule, and Simpson's Three-Eighths Rule. The output includes the calculated integration values, offering users insights into the accuracy and suitability of each method for different functions and intervals.
 now write latex code for this with full package support
\end{document}
```

This LaTeX code includes package support for geometry, graphicx, listings, amsmath, and float to enhance the formatting and layout of the lab report. Adjust the filenames in the `\includegraphics` commands to match the actual filenames of your screenshots.
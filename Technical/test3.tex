% % \documentclass[11pt,a4paper]{article}
% % \usepackage{ulem}
% % \usepackage{color}
% % \begin{document}
% % 1. {\em Normal emphasized texts.}
% % \ULforem % Redefining effects of {\em } and \emph{}.
% % 2. {\em Underlined texts with line breaks.}
% % 3. \emph{Yet underlined texts with line breaks.}
% % 4. {\it{\em Underlined texts in italic fonts …}}
% % 5. \textit{\emph{Yet underlined texts in …}}
% % \normalem % For normal effects of {\em } and \emph{}.
% % 6. {\em Returned to normal emphasized texts.}
% % 7. The next phase is \uwave{wavy underlined.}
% % 8. The next phase is \sout{striked out statement.}
% % 9. The next phase is \xout{crossed out statement.}

% % \noindent
% % \texttt{atext} or {\tt atext}\\
% % \textsf{atext} or {\sf atext}\\
% % \textrm{atext} or {\rm\bf atext}\\{\bf atext}\\
% % \textup{\it{atext}} or {\upshape atext}\\
% % \underline{atext}\\
% % \textsc{atext} or {\sc atext}\\
% % \textsl{atext} or {\sl atext}\\
% % \textmd{atext} or {\mdseries atext}\\
% % \textbf{atext} or {\bfseries atext}\\
% % \ULforem
% % \emph{atext} or {\em atext}\\
% % \definecolor{red}{rgb}{1,0,0}
% % \textcolor{red}{atext}\\
% % \textcolor{blue}{atext}\\
% % \textcolor{green}{atext}\\
% % \textcolor{cyan}{atext}\\
% % \textcolor{magenta}{atext}\\
% % \textcolor{yellow}{atext}\\
% % \textcolor{black}{atext}\\
% % %\textcolor{white}{atext}\\
% % \textcolor{red}{\textbf{atext}}\\
% % \textcolor{red}{\textit{atext}}\\
% % there is a big \textcolor{red}{\bf atext}
% % \end{document}
% % \documentclass{article}
% % \usepackage{color}
% % \usepackage{hyperref}
% % \begin{document}
% % \noindent
% % \underline{\bf LaTeX} is a macro package for typesetting documents. It
% % is a language-based approach, where LaTeX instructions
% % are interspersed with \textcolor{red}{the} text file of a document, say
% % myfile.tex, for obtaining the desired output as
% % myfile.dvi. The \underline{myfile.dvi} file can then be used to
% % generate myfile.ps or myfile.pdf file.\\
% % \section{Centre of gravity}{label:sec:cg}
% % This is the point though which the resultant
% % of the gravitational forces of all elemental
% % weights of a body acts.
% % \subsection{First subsection}
% % This is the point though which the resultant
% % of the gravitational forces of all elemental
% % weights of a body acts.
% % \par
% % (1) There are certain key issues to attract
% % investors, which need to be addressed.\ref{sec:cg}
% % \begin{center}
% %     \underline{\textbf{Ans the question!!}}
% % \end{center}
% % \begin{flushleft}
% %     Question (1) is answered here.\\
% %     Question (2) is answered here.
% % \end{flushleft}
% % \begin{center}
% %     Question (3) is answered here.\\
% %     Question (4) is answered here.
% % \end{center}
% % \noindent
% % \paragraph{1. Paragraph} This is the first paragraph.
% % \subparagraph{1.1 Sentence in Succeeding Paragraphs} This is the first sentence in succeeding paragraphs.

% % \begin{sloppypar}naimish\end{sloppypar}
% % \end{document}
% % \documentclass{book}
% % \begin{document}
% % \chapter{\underline{First chapter}}
% % \section{Centre of gravity}\label{sec:cg}
% % This is the point though which the resultant
% % of the gravitational forces of all elemental
% % weights of a body acts.
% % \subsection{\underline{First subsection}}
% % This is the point though which the resultant
% % of the gravitational forces of all elemental
% % weights of a body acts.
% % \subsubsection{First subsubsection}
% % This is the point though which the resultant
% % \par
% % (1) There are certain key issues to attract
% % investors, which need to be addressed.
% % \chapter{Second chapter}
% % \section{Centre of gravity}\label{sec:cg2}
% % This is the point though which the resultant
% % of the gravitational forces of all elemental
% % weights of a body acts.
% % \begin{center}
% %     \LaTeX\ prints texts with both side aligned.
% %     Center aligned texts can …
% % \end{center}
% % \end{document}
% % \documentclass[12pt,a4paper,twoside]{book}
% %  \usepackage{fancyheadings} % 
% %  \pagestyle{fancy} 
% %  \renewcommand{\chaptermark}[1]{\markboth{\thechapter. #1}{}}
% % \renewcommand{\sectionmark}[1]{\markright{\thesection. #1}}
% % \lhead[\textbf{\thepage}]{\textbf{\rightmark}} 
% % \rhead[\textbf{\leftmark}]{\textbf{\thepage}} 
% % \lfoot[\textbf{Engineering Mechanics}]{}
% %  \rfoot[]{\textbf{Dilip Datta}}
% %  \cfoot[]{}
% %  \renewcommand{\headrulewidth}{0.15mm}
% %  \renewcommand{\footrulewidth}{0.15mm}
% %  \addtolength{\headwidth}{\marginparsep} \addtolength{\headwidth}{\marginparwidth}
% %  %
% % \begin{document}

% % \chapter{Distributed Force System}
% % \thispagestyle{empty}
% % For simplifying an analysis, the force exerted by a body…
% % …
% % \section{Centre of Gravity} Since the weight of a body is a system of concurrent forces…
% % …
% % \begin{enumerate}
% %     \item Question 1
% %     \item Question 2
% %     \item Question 3
% %     \begin{enumerate}
% %         \item Question 1
% %          \begin{enumerate}
% %             \item Question 1
% %             \begin{enumerate}
% %                 \item Question 1
% %                 \item Question 2
% %             \end{enumerate}
% %             \item Question 2
% %             \begin{enumerate}
% %                 \item Question 1
% %             \end{enumerate}
% %         \end{enumerate}
% %         \item Question 2
% %     \end{enumerate}
% %     \item Question 4
% % \end{enumerate}
% % \end{document}
\documentclass[a4paper,11pt]{article}
\usepackage{enumerate}
\usepackage{color}
\usepackage{ulem}
\usepackage{multicolrule}

%
\begin{document}
\noindent
\underline{\bf Lorem ipsum dolor sit amet}\vspace{4mm}

elit ut aliquam purus. Purus gravida quis blandit turpis cursus. 
Eu scelerisque felis imperdiet proin. 
\xout{Sed risus ultricies tristique nulla} aliquet enim tortor at. 
Proin sagittis nisl rhoncus mattis rhoncus urna neque viverra. 
Adipiscing at in tellus integer feugiat.\\ 

{\bf Tristique senectus et netus et:} malesuada fames ac turpis egestas.
Cursus mattis \textcolor{red}{molestie} a iaculis at erat pellentesque adipiscing commodo. 
Ac auctor augue mauris augue neque gravida.Nullam vehicula ipsum a arcu cursus. 
Tincidunt nunc pulvinar sapien et ligula ullamcorper malesuada proin libero. 
Id diam vel quam elementum. Maecenas pharetra convallis posuere morbi leo urna molestie at elementum. 
Neque viverra justo nec ultrices dui sapien eget mi.
\begin{itemize}

\item Dolor sit amet consectetur adipiscing .
\item labore et dolore magna aliqua.
\item Urna nunc id cursus metus aliquam eleifend mi. 
\end{itemize}
\end{document}

% \documentclass[twocolumn]{article}
% \usepackage{multicol}
% \columnseprule = 1mm
% %
% \begin{document}
% The standard document-classes permit
% to print a document either in a …
% \end{document}
% \documentclass[11pt,a4paper]{article}
% \usepackage{enumerate}
% %
% \begin{document}
% \begin{center}{\bf EXAMPLES}\end{center}
% \begin{enumerate}[{\bf Ex{a}mple 1:}]
% \item Show that…
% \item Prove that…\label{item:ex_gr}
% \item What would be…
% \end{enumerate}
% %
% \begin{center}{\bf PROBLEMS}\end{center}
% \begin{enumerate}[{\bf Problem (a):}]
% \item Prove that…\label{item:pr_gr}
% \item Show that…
% \item What would be…
% \end{enumerate}
% %
% The problem (\ref{item:pr_gr}) is just an
% extension of the example \ref{item:ex_gr}.
% \end{document}
% \documentclass[a5paper, landscape]{book}
% \usepackage[utf8]{inputenc}
% \usepackage[T1]{fontenc}
% \usepackage{geometry}
% \usepackage{lipsum}
% \usepackage{titlesec}

% \geometry{
%     paper=a5paper,
%     top=2cm,
%     bottom=2cm,
%     left=3cm,
%     right=3cm,
%     includefoot
% }
% \pagenumbering{roman}

% \begin{document}
% \chapter{First Chapter}
% \section{Section One}
% \lipsum[1-2]

% \chapter{Second Chapter}
% \section{Section Two}
% \lipsum[3-4]

% \end{document}

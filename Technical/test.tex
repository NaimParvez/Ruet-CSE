\documentclass[a4paper]{book}

\usepackage{hyperref}
%\usepackage{boxedminipage}


\begin{document}
\chapter{First chapter}
\section{Centre of gravity}\label{sec:cg}
This is the point though which the resultant
of the gravitational forces of all elemental
weights of a body acts.
\subsection{First subsection}
This is the point though which the resultant
of the gravitational forces of all elemental
weights of a body acts.
\par
(1) There are certain key issues to attract
investors, which need to be addressed.
%
\par \parindent = 8mm
(2) There are certain key issues to attract
investors, which need to be addressed.
%
\par \noindent \parskip 2mm
(3) There are certain key issues to attract
investors, which need to be addressed.
\par
(4) There are certain key issues to attract
investors, which need to be addressed.
%
\paragraph{(5) Investing policies:}
There are certain key issues to attract
investors, which need to be addressed.
%
\subparagraph{(6) Investing policies:}
There are certain key issues to attract
investors, which need to be addressed.
\section{Centre of mass}\label{sec-ex}
The definition of the centre of gravity is
given in Section\ref{sec:cg2} …
\chapter{Second chapter}
\section{Centre of gravity}\label{sec:cg2}
This is the point though which the resultant
of the gravitational forces of all elemental
weights of a body acts.
\begin{center}
    \LaTeX\ prints texts with both side aligned.
    Center aligned texts can …
\end{center}
\begin{flushright}
    \LaTeX\ prints texts with both side aligned.
    Right aligned texts can …
\end{flushright}
\begin{flushleft}
    \LaTeX\ prints texts with both side aligned.
    Left aligned texts can be produced through
    the ‘flushleft’ environment.
\end{flushleft}
\begin{quotation}
% \begin{spacing}{1.2}
 \noindent
Quoted statements are also printed with both side
aligned, but in a narrowed width.
\begin{flushright}
{\it - Anonymous}
\end{flushright}
% \end{spacing}
\end{quotation}{spacing}
The ‘quotation’ environment is used for printing
quoted \begin{sloppypar}statements\end{sloppypar} in a \textbf{`narrowed'} -- width.\\
% \begin{boxedminipage}[t]{0.4\linewidth}
%     This boxed-minipage is also …
% \end{boxedminipage}


Marks: 100 \hfill Time: 3 Hours
\begin{flushright}
    \textbf{[Maximum marks: 100]}
\end{flushright}
Both Rubi and Lila\footnote{They are sisters.} study
in class I, while Ravi and Joy\footnote{They are
friends.\label{fn:friends}} study in class II.
\end{document}
